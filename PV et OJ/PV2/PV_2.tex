%!TEX encoding = IsoLatin

%

%--------------------------------------------------------------------------------------
%------------------------------------- preambule --------------------------------------
%--------------------------------------------------------------------------------------
\documentclass[12pt]{ULojpv}

% Chargement des packages supplementaires
\usepackage[french]{babel}
\usepackage[utf8]{inputenc}
%\usepackage[ansinew]{inputenc}
\usepackage{pifont}


% Definitions des parametres de l'en-tete
\Cours{GPH-3110 Projet de conception en génie physique}             % Nom du cours
\NumeroEquipe{2}                                     % Numero de l'equipe
\NomEquipe{MesurOptic}                               % Nom de l'equipe
\Objet{Procès-verbal}                                 % Nom du document
\SujetRencontre{Organisation et planification}             % Sujet de la rencontre
\DateRencontre{30 Janvier 2023}                            % Date de la rencontre
\LocalRencontre{PLT-2701}                            % Local de la rencontre
\HeureRencontre{15h39}                          % Heure de la rencontre


%--------------------------------------------------------------------------------------
%--------------------------------- corps du document ----------------------------------
%--------------------------------------------------------------------------------------
\begin{document}
\entete
\begin{enumerate}

% nouveau point
\item \textbf{Ouverture de la réunion}

Heure: 15h30

\item \textbf{Étaient présents}

\begin{dinglist}{"33}
   \item Sammy Noël Parisé
   \item Alexandra Alain-Beaudoin
   \item Benjamin Trudel
   \item Antoine Gagnon
   \item Louis Cormier
   \item Philippe Truchon

\end{dinglist}

\item \textbf{Étaient absents}

% nouveau point
\item \textbf{Affaires découlant du procès-verbal de la dernière rencontre}

Aucun point n'a été soulevé.


% nouveau point
\item \textbf{Points à traiter}
   \begin{enumerate}
      \item Discussion et détermination de l'échéancier
      
      L'échéancier produit par le gestionnaire est approuvé par l'équipe.
      L'équipe recommande l'ajout des commandes des lentilles et de l'électronique. 
      
      
      \item Discussion sur le design préliminaire du système
      
      Suite à la discussion avec Pr. Bernier, nous décidons d'utiliser une lentille grossissante pour le système.
      L'ensemble de l'absorbtion de la lumière se fera sur le filtre final là où se trouvera les thermistances.
      
      \item Discussion sur la répartition des tâches
      
      L'équipe manque de temps pour effectuer la répartition des tâches exactes. Toutefois il est convenu qu'une
      seconde réunion en milieu de semaine sera nécessaire. La date sera discutée dans le groupe messenger.
      
      
   \end{enumerate}

% nouveau point
\item \textbf{Divers}
   Aucun point divers n'a été soulevé.

% nouveau point
\item \textbf{Répartition des tâches}
   Aucune répartition des tâches n'a été effectuée.


% nouveau point

% nouveau point
\item \textbf{Date, heure, lieu et objectif de la prochaine réunion}

\begin{tabular}{@{}lll}
   Date indéterminée (à confirmer, voir groupe messenger)
\end{tabular}
\par

Objectif: Établir la répartition des tâches, principalement pour la partie simulation thermique/optique


% nouveau point
\item \textbf{Fermeture de la réunion}

Heure: 16h00


% nouveau point


\end{enumerate}

% align to right
\begin{flushright}
   \textbf{Le gestionnaire de projet, Sammy, le 30 janvier 2023}

\end{flushright}

\end{document}

