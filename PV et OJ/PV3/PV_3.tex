%!TEX encoding = IsoLatin

%

%--------------------------------------------------------------------------------------
%------------------------------------- preambule --------------------------------------
%--------------------------------------------------------------------------------------
\documentclass[12pt]{ULojpv}

% Chargement des packages supplementaires
\usepackage[french]{babel}
\usepackage[utf8]{inputenc}
%\usepackage[ansinew]{inputenc}
\usepackage{pifont}


% Definitions des parametres de l'en-tete
\Cours{GPH-3110 Projet de conception en génie physique}             % Nom du cours
\NumeroEquipe{2}                                     % Numero de l'equipe
\NomEquipe{MesurOptic}                               % Nom de l'equipe
\Objet{Procès-verbal}                                 % Nom du document
\SujetRencontre{Organisation et planification}             % Sujet de la rencontre
\DateRencontre{2 Février 2023}                            % Date de la rencontre
\LocalRencontre{À distance}                            % Local de la rencontre
\HeureRencontre{19h00}                          % Heure de la rencontre


%--------------------------------------------------------------------------------------
%--------------------------------- corps du document ----------------------------------
%--------------------------------------------------------------------------------------
\begin{document}
\entete
\begin{enumerate}

% nouveau point
\item \textbf{Ouverture de la réunion}

Heure: 19h00

\item \textbf{Étaient présents}

\begin{dinglist}{"33}
   \item Sammy Noël Parisé
   \item Alexandra Alain-Beaudoin
   \item Benjamin Trudel
   \item Louis Cormier
   \item Philippe Truchon

\end{dinglist}

\item \textbf{S'étaient excusés}
\begin{dinglist}{"33}
   \item Antoine Gagnon

\end{dinglist}

% nouveau point
\item \textbf{Affaires découlant du procès-verbal de la dernière rencontre}

Aucun point n'a été soulevé.


% nouveau point
\item \textbf{Points à traiter}
   \begin{enumerate}
      \item Discussion de l'avancement durant les trois derniers jours\\
      Philippe a finalisé le design du circuit de mesure de la température. Des questions
      émergent quant à la précision du circuit. Il est possible qu'il y ait du "coupling" entre 
      les themistances. Des tests sur breadboards seront effectués pour confirmer ou infirmer cette 
      hypothèse.\\

      Alexandra a identifié les composants nécessaires pour la partie optique. Il reste à
      envoyer les infos à Louis pour qu'il puisse faire les simulations thermiques. \\
      
      Louis les a déjà entâmées, il ne reste qu'à choisir les bonnes valeurs des différents coefficients.
      Les simulations préliminaires indiquent du potentiel pour le design choisi.\\

      
   \end{enumerate}

% nouveau point
\item \textbf{Divers}
   Aucun point divers n'a été soulevé.

% nouveau point
\item \textbf{Répartition des tâches}
   Chacun des membres continue à travailler sur leur tâche respective.\\
   Sammy fera un tour des différentes tâches pour s'assurer que tout le monde
   est sur la bonne voie et donner un coup de main.


% nouveau point

% nouveau point
\item \textbf{Date, heure, lieu et objectif de la prochaine réunion}

\begin{tabular}{@{}lll}
   \textbf{Date:} & 6 février 2023 & \\
   \textbf{Heure:} & 14h30 & \\
   \textbf{Lieu:} & PLT-2701 & \\
\end{tabular}
\par

Objectif: Confirmer le design du circuit de mesure de température 
            et établir un échéancier pour finaliser le design préliminaire au courant de la semaine.


% nouveau point
\item \textbf{Fermeture de la réunion}

Heure: 19h45


% nouveau point


\end{enumerate}

% align to right
\begin{flushright}
   \textbf{Le gestionnaire de projet, Sammy, le 2 février 2023}

\end{flushright}

\end{document}

