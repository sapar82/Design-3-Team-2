

%
% Exemple d'ordre du jour
% par Pierre Tremblay, Universite Laval
% modifie par Christian Gagne, Universite Laval
% 2011/01/14 - version 1.4
% modifié par Robert Bergevin, Université Laval
% 24/11/2011
% modifié par Jean-Yves Chouinard, Université Laval
% 2016/01/11
% modifié par Jean-Yves Chouinard, Université Laval
% 2017/01/04
%

%--------------------------------------------------------------------------------------
%------------------------------------- preambule --------------------------------------
%--------------------------------------------------------------------------------------
\documentclass[12pt]{ULojpv}

% Chargement des packages supplementaires
%\usepackage[ansinew]{inputenc}
\usepackage[french]{babel}
\usepackage[utf8]{inputenc}
%\usepackage{utf8}

% Definitions des parametres de l'en-tete
\Cours{GPH-3110 Conception en génie physique}             % Nom du cours
\NumeroEquipe{XX}                                     % Numero de l'equipe
\NomEquipe{XX}                          % Nom de l'equipe
\Objet{Ordre du jour}                                 % Nom du document
\SujetRencontre{Organisation et planification}        % Sujet de la rencontre
\DateRencontre{2021/02/04}                            % Date de la rencontre
\LocalRencontre{XXX}                            % Local de la rencontre
\HeureRencontre{XXXX}                          % Heure de la rencontre

%--------------------------------------------------------------------------------------
%--------------------------------- corps du document ----------------------------------
%--------------------------------------------------------------------------------------
\begin{document}
\entete
\begin{enumerate}
   \item \textbf{Ouverture de la réunion}
   \item \textbf{Lecture de l'ordre du jour}
   %\item \textbf{Lecture et adoption du procès-verbal de la réunion du 3 février 2021}
   \item \textbf{Affaires découlant du procès-verbal}
   \item \textbf{Retour sur les tâches effectuées}
      \begin{enumerate}
         \item X
         \item 2i2i
      \end{enumerate}
%      \begin{enumerate}
%         \item Affaire \#1
%            \begin{enumerate}
%               \item Sous-affaire \#1
%               \item Sous-affaire \#2
%            \end{enumerate}
%         \item Affaire \#2
%      \end{enumerate}
   \item \textbf{Points à traiter}
      \begin{enumerate}
         \item Discussion au sujet des séminaires techniques
            
      \end{enumerate}
   \item \textbf{Divers}
   \item \textbf{Répartition des tâches}
   \item \textbf{Évaluation de la réunion}

   \item \textbf{Fermeture de la réunion}
\end{enumerate}

\end{document}

