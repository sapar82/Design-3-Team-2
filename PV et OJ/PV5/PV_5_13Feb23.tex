


%--------------------------------------------------------------------------------------
%------------------------------------- preambule --------------------------------------
%--------------------------------------------------------------------------------------
\documentclass[12pt]{ULojpv}

% Chargement des packages supplementaires
%\usepackage[ansinew]{inputenc}
\usepackage[french]{babel}
\usepackage[utf8]{inputenc}
%\usepackage{utf8}

% Definitions des parametres de l'en-tete
\Cours{GPH-3110 Conception en génie physique}             % Nom du cours
\NumeroEquipe{2}                                     % Numero de l'equipe
\NomEquipe{Scavengers}                          % Nom de l'equipe
\Objet{Procès-Verbal}                                 % Nom du document
\SujetRencontre{Design préliminaire}        % Sujet de la rencontre
\DateRencontre{13 Février 2023, 14h30}                            % Date de la rencontre
\LocalRencontre{PLT-2708}                            % Local de la rencontre
\DureeRencontre{2 heures}                          % Heure de la rencontre

%--------------------------------------------------------------------------------------
%--------------------------------- corps du document ----------------------------------
%--------------------------------------------------------------------------------------
\begin{document}
\entete

\textbf{Étaient présents}

Tous les membres étaient présents.

\textbf{Étaient absents}

Aucun membre de l'équipe n'était absent.

\begin{enumerate}
   \item \textbf{Ouverture de la réunion}

   La réunion est ouverte à 14h00 puisque l'ensemble des membres sont arrivés à l'heure.
   \item \textbf{Lecture de l'ordre du jour}
   \item \textbf{Suivi des tâches}
   
   L'ensemble des tâches ont été complétées à temps.

   \item \textbf{Points à traiter}
      \begin{enumerate}
         \item Discussion des modifications à effectuer au design préliminaire

         Deux designs mécaniques sont proposés. Le premier nécessite l'assemblage des
         thermistances au préalable à l'assemblage du bloc les contenant et le second 
         nécessite l'assemblage du bloc avant l'assemblage des thermistances. L'équipe 
         ne s'entend pas sur le design à choisir. Le consensus est qu'il faudra effectuer
         des tests pour déterminer lequel des deux designs est le plus efficace.

         \item Discussion sur les plans de tests à effectuer
         
         Les plans de test proposés par l'équipe conviennent à tous. L'échéancier
         exact de la réalisation de ces tests sera déterminé lors de la prochaine
         réunion.

         \item Discussion sur le rapport préliminaire à remettre
         
         L'équipe s'accorde sur la structure des différents composants du rapport préliminaire.

      \end{enumerate}

   \item \textbf{Divers}

      Suite à la discussion avec Martin, l'équipe décide de commander l'électronique au courant de la semaine.
   
   \item \textbf{Répartition des tâches}
      \begin{itemize}
         \item \textbf{Sammy} : Écriture du rapport, commande de l'électronique et réalisation d'un circuit sur breadboard.
         \item \textbf{Antoine} : Finalisation du design préliminaire et écriture du rapport.
         \item \textbf{Benjamin} : Écriture du rapport.
         \item \textbf{Louis} : Réalisation des simulations avec radiation et écriture du rapport.
         \item \textbf{Alexandra} : Recherche d'informations sur le verre absorbant et écriture du rapport.
         \item \textbf{Philippe} : Commande des PCB et écriture du rapport.
      \end{itemize}
   
   \item \textbf{Planification de la prochaine réunion}
   
   La prochaine réunion aura lieu le 20 février 2023 à 14h00 au PLT-2708.

   \item \textbf{Fermeture de la réunion}
   Heure: 16h30
\end{enumerate}

\begin{flushright}
   \textbf{Le gestionnaire de projet, Sammy, le 2023-02-13}

\end{flushright}

\end{document}

