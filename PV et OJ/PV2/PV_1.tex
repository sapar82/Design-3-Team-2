%!TEX encoding = IsoLatin

%

%--------------------------------------------------------------------------------------
%------------------------------------- preambule --------------------------------------
%--------------------------------------------------------------------------------------
\documentclass[12pt]{ULojpv}

% Chargement des packages supplementaires
\usepackage[french]{babel}
\usepackage[utf8]{inputenc}
%\usepackage[ansinew]{inputenc}
\usepackage{pifont}


% Definitions des parametres de l'en-tete
\Cours{GPH-3110 Projet de conception en génie physique}             % Nom du cours
\NumeroEquipe{3}                                     % Numero de l'equipe
\NomEquipe{XXX}                               % Nom de l'equipe
\Objet{Procès-verbal}                                 % Nom du document
\SujetRencontre{Organisation et planification}             % Sujet de la rencontre
\DateRencontre{23 Janvier 2023}                            % Date de la rencontre
\LocalRencontre{PLT-2701}                            % Local de la rencontre
\HeureRencontre{15h39}                          % Heure de la rencontre


%--------------------------------------------------------------------------------------
%--------------------------------- corps du document ----------------------------------
%--------------------------------------------------------------------------------------
\begin{document}
\entete
\begin{enumerate}

% nouveau point
\item \textbf{Ouverture de la réunion}

Heure: 15h30

\item \textbf{Étaient présents}

\begin{dinglist}{"33}
   \item Sammy Noël Parisé
   \item Alexandra Alain-Beaudoin
   \item Benjamin Trudel
   \item Antoine Gagnon
   \item Louis Cormier
   \item Philippe Truchon

\end{dinglist}

\item \textbf{Étaient absents}

% nouveau point
\item \textbf{Affaires découlant du procès-verbal de la dernière rencontre}
   Aucun procès verbal n'avait été rédigé suite à la dernière réunion.


% nouveau point
\item \textbf{Points à traiter}
   \begin{enumerate}
      \item Discussion sur le design préliminaire du système
      
      L'équipe s'accorde pour explorer la possibilité de créer un détecteur qui utiliserait une matrice de 
      themistors collés sur une lentille pour détecter la température de la surface de la lentille. 
      
      Deux designs sont explorés soit en croix et en quadrillés. L'équipe s'accorde pour explorer la possibilité d'utiliser un ensembl
      de filtres pour diminuer la puissance de la lumière incidente sur la lentille. 

      
      Une paire de lentille permettant le grossissement du faisceau laser est également envisagée.

      \item Discussion sur l'échéancier
      
      L'équipe s'accorde pour avoir terminé l'ensemble des simulation à la quatrième semaine soit le 5 Février.
      L'échéancier complet sera proposé lors de la prochaine réunion par le gestionnaire de projet.

      \item Discussion sur la répartition des tâches
      
      Comme il y a peu de tâches de conception mécanique pour le moment, Antoine participera aux simulation thermiques avec Louis.
      Les autres membres de l'équipe poursuivront leurs tâches respectives.
      
      
   \end{enumerate}

% nouveau point
\item \textbf{Divers}
   Aucun point divers n'a été soulevé.

% nouveau point
\item \textbf{Répartition des tâches}
   \begin{enumerate}
      \item Louis et Antoine s'occupent de la simulation de l'absorbtion de la lumière par la lentille. Un ensemble préliminaire
      de simulations seront présentés lors de la prochaine réunion le 30 Janvier.

      \item Philippe effectura des recherches sur la taille des thermistors et l'arrangement que ceux-ci pourront prendre.
      
      \item Alexandra magasinera les composantes nécessaires pour produire une paire de lentilles grossissantes.

      \item  Benjamin continue à travailler sur l'interface graphique dans Labview.

      \item Sammy poursuit la rédaction des OJ, PVs et autres documents nécessaires au projet.
   \end{enumerate}


% nouveau point

% nouveau point
\item \textbf{Date, heure, lieu et objectif de la prochaine réunion}

\begin{tabular}{@{}lll}
   Date: 2023-01-30
   & Heure: 15h30
   &  Lieu: PLT-2701
\end{tabular}
\par

Objectif: Présentation des simulations thermiques et des possibilités de design de la lentille. Détermination de l'échéancier complet 
et du design préliminaire avant les simulations finales. 


% nouveau point
\item \textbf{Fermeture de la réunion}

Heure: 16h30


% nouveau point


\end{enumerate}

% align to right
\begin{flushright}
   \textbf{Le gestionnaire de projet, Sammy}

\end{flushright}

\end{document}

