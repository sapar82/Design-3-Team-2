%!TEX encoding = IsoLatin

%
% Exemple de procès-verbal
% par Pierre Tremblay, Universite Laval
% modifie par Christian Gagne, Universite Laval
% % 2011/01/14 - version 1.4
% modifié par Robert Bergevin, Université Laval
% 24/11/2011
% modifié par Jean-Yves Chouinard, Université Laval
% 2016/01/11
% modifié par Jean-Yves Chouinard, Université Laval
% 2017/01/04
%

%--------------------------------------------------------------------------------------
%------------------------------------- preambule --------------------------------------
%--------------------------------------------------------------------------------------
\documentclass[12pt]{ULojpv}

% Chargement des packages supplementaires
\usepackage[french]{babel}
\usepackage[utf8]{inputenc}
%\usepackage[ansinew]{inputenc}
\usepackage{pifont}


% Definitions des parametres de l'en-tete
\Cours{GPH-3110 Projet de conception en génie physique}             % Nom du cours
\NumeroEquipe{XX}                                     % Numero de l'equipe
\NomEquipe{XXX}                               % Nom de l'equipe
\Objet{Procès-verbal}                                 % Nom du document
\SujetRencontre{Organisation et planification}             % Sujet de la rencontre
\DateRencontre{XXX}                            % Date de la rencontre
\LocalRencontre{XX}                            % Local de la rencontre
\HeureRencontre{XXX}                          % Heure de la rencontre


%--------------------------------------------------------------------------------------
%--------------------------------- corps du document ----------------------------------
%--------------------------------------------------------------------------------------
\begin{document}
\entete
\begin{enumerate}

% nouveau point
\item \textbf{Ouverture de la réunion}

Heure: xxxx

\item \textbf{Étaient présents}

\begin{dinglist}{"33}
   \item Sammy Noël Parisé

\end{dinglist}

\item \textbf{Étaient absents}



% nouveau point

\item \textbf{Lecture du procès-verbal de la réunion }


% nouveau point
\item \textbf{Affaires découlant du procès-verbal}


% nouveau point
\item \textbf{Points à traiter}


% nouveau point
\item \textbf{Divers}

% nouveau point
\item \textbf{Répartition des tâches}


% nouveau point

% nouveau point
\item \textbf{Date, heure, lieu et objectif de la prochaine réunion}

\begin{tabular}{@{}lll}
   Date: 2021/2/13
   & Heure: 10h30
   &  Lieu: À distance
\end{tabular}
\par
Description de l'objectif ou des objectifs.


% nouveau point
\item \textbf{Fermeture de la réunion}

Heure: 12h21


% nouveau point


\end{enumerate}

% align to right
\begin{flushright}
   \textbf{Le président}

\end{flushright}

\end{document}

